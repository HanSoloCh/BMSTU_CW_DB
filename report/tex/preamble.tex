\usepackage{listings} % Листинги
\usepackage{xcolor} % Цвета

% \usepackage{listingsutf8}
% Для листинга кода:
\lstset{%
	language=c++,   					% выбор языка для подсветки	
	basicstyle=\small\sffamily,			% размер и начертание шрифта для подсветки кода
	numbersep=5pt,
	numbers=left,						% где поставить нумерацию строк (слева\справа)
	%numberstyle=,					    % размер шрифта для номеров строк
	stepnumber=1,						% размер шага между двумя номерами строк
	xleftmargin=17pt,
	showstringspaces=false,
	numbersep=5pt,						% как далеко отстоят номера строк от подсвечиваемого кода
	frame=single,						% рисовать рамку вокруг кода
	tabsize=4,							% размер табуляции по умолчанию равен 4 пробелам
	captionpos=t,						% позиция заголовка вверху [t] или внизу [b]
	breaklines=true,					
	breakatwhitespace=true,				% переносить строки только если есть пробел
	escapeinside={\#*}{*)},				% если нужно добавить комментарии в коде
	backgroundcolor=\color{white}
}