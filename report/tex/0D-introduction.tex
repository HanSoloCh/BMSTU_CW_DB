\begin{center}
    \textbf{ВВЕДЕНИЕ}
\end{center}
\addcontentsline{toc}{chapter}{ВВЕДЕНИЕ}

Издревле библиотеки служили важнейшим источником информации. До появления интернета, а также в эпоху его раннего развития, именно библиотеки часто были единственным местом, где можно было найти ответ на интересующий вопрос.

Однако и в цифровую эпоху, когда в интернете доступно огромное количество информации, библиотеки остаются ценным ресурсом. Они востребованы как людьми, предпочитающими работать с физическими носителями, так и теми, кому нужна информация, которую невозможно найти в оцифрованном виде. Кроме того, библиотеки предоставляют доступ не только к книгам, но и к газетам, журналам, плакатам, дискам и другим печатным и электронным материалам.

Появление интернета не уничтожило библиотеки, но существенно изменило их работу. Теперь нет необходимости хранить в бумажном виде информацию об изданиях, имеющихся в наличии, читателях и выдачах книг: все это переносится в цифровой формат, где информация становится проще в обработке и доступе.

Целью данной курсовой работы является разработка базы данных для хранения и обработки данных библиотечной картотеки.

Для достижения поставленной цели необходимо решить следующие задачи:

\begin{itemize}
	\item[---] провести анализ существующих библиотечных сервисов;
	\item[---] определить требования к базе данных и программному обеспечению;
	\item[---] спроектировать сущности базы данных и задать их ограничения;  
	\item[---] выбрать инструменты для реализации базы данных и программного обеспечения;  
	\item[---] реализовать базу данных и программное обеспечение для работы с ней;  
	\item[---] провести исследование на основе разработанной базы данных.
\end{itemize}	
 